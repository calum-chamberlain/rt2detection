% Generated by Sphinx.
\def\sphinxdocclass{report}
\documentclass[letterpaper,10pt,english]{sphinxmanual}
\usepackage[utf8]{inputenc}
\DeclareUnicodeCharacter{00A0}{\nobreakspace}
\usepackage{cmap}
\usepackage[T1]{fontenc}
\usepackage{babel}
\usepackage{times}
\usepackage[Bjarne]{fncychap}
\usepackage{longtable}
\usepackage{sphinx}
\usepackage{multirow}

\addto\captionsenglish{\renewcommand{\figurename}{Fig. }}
\addto\captionsenglish{\renewcommand{\tablename}{Table }}
\floatname{literal-block}{Listing }



\title{rt2detection Documentation}
\date{April 22, 2015}
\release{0.0.1}
\author{Calum John Chamberlain}
\newcommand{\sphinxlogo}{}
\renewcommand{\releasename}{Release}
\makeindex

\makeatletter
\def\PYG@reset{\let\PYG@it=\relax \let\PYG@bf=\relax%
    \let\PYG@ul=\relax \let\PYG@tc=\relax%
    \let\PYG@bc=\relax \let\PYG@ff=\relax}
\def\PYG@tok#1{\csname PYG@tok@#1\endcsname}
\def\PYG@toks#1+{\ifx\relax#1\empty\else%
    \PYG@tok{#1}\expandafter\PYG@toks\fi}
\def\PYG@do#1{\PYG@bc{\PYG@tc{\PYG@ul{%
    \PYG@it{\PYG@bf{\PYG@ff{#1}}}}}}}
\def\PYG#1#2{\PYG@reset\PYG@toks#1+\relax+\PYG@do{#2}}

\expandafter\def\csname PYG@tok@gd\endcsname{\def\PYG@tc##1{\textcolor[rgb]{0.63,0.00,0.00}{##1}}}
\expandafter\def\csname PYG@tok@gu\endcsname{\let\PYG@bf=\textbf\def\PYG@tc##1{\textcolor[rgb]{0.50,0.00,0.50}{##1}}}
\expandafter\def\csname PYG@tok@gt\endcsname{\def\PYG@tc##1{\textcolor[rgb]{0.00,0.27,0.87}{##1}}}
\expandafter\def\csname PYG@tok@gs\endcsname{\let\PYG@bf=\textbf}
\expandafter\def\csname PYG@tok@gr\endcsname{\def\PYG@tc##1{\textcolor[rgb]{1.00,0.00,0.00}{##1}}}
\expandafter\def\csname PYG@tok@cm\endcsname{\let\PYG@it=\textit\def\PYG@tc##1{\textcolor[rgb]{0.25,0.50,0.56}{##1}}}
\expandafter\def\csname PYG@tok@vg\endcsname{\def\PYG@tc##1{\textcolor[rgb]{0.73,0.38,0.84}{##1}}}
\expandafter\def\csname PYG@tok@m\endcsname{\def\PYG@tc##1{\textcolor[rgb]{0.13,0.50,0.31}{##1}}}
\expandafter\def\csname PYG@tok@mh\endcsname{\def\PYG@tc##1{\textcolor[rgb]{0.13,0.50,0.31}{##1}}}
\expandafter\def\csname PYG@tok@cs\endcsname{\def\PYG@tc##1{\textcolor[rgb]{0.25,0.50,0.56}{##1}}\def\PYG@bc##1{\setlength{\fboxsep}{0pt}\colorbox[rgb]{1.00,0.94,0.94}{\strut ##1}}}
\expandafter\def\csname PYG@tok@ge\endcsname{\let\PYG@it=\textit}
\expandafter\def\csname PYG@tok@vc\endcsname{\def\PYG@tc##1{\textcolor[rgb]{0.73,0.38,0.84}{##1}}}
\expandafter\def\csname PYG@tok@il\endcsname{\def\PYG@tc##1{\textcolor[rgb]{0.13,0.50,0.31}{##1}}}
\expandafter\def\csname PYG@tok@go\endcsname{\def\PYG@tc##1{\textcolor[rgb]{0.20,0.20,0.20}{##1}}}
\expandafter\def\csname PYG@tok@cp\endcsname{\def\PYG@tc##1{\textcolor[rgb]{0.00,0.44,0.13}{##1}}}
\expandafter\def\csname PYG@tok@gi\endcsname{\def\PYG@tc##1{\textcolor[rgb]{0.00,0.63,0.00}{##1}}}
\expandafter\def\csname PYG@tok@gh\endcsname{\let\PYG@bf=\textbf\def\PYG@tc##1{\textcolor[rgb]{0.00,0.00,0.50}{##1}}}
\expandafter\def\csname PYG@tok@ni\endcsname{\let\PYG@bf=\textbf\def\PYG@tc##1{\textcolor[rgb]{0.84,0.33,0.22}{##1}}}
\expandafter\def\csname PYG@tok@nl\endcsname{\let\PYG@bf=\textbf\def\PYG@tc##1{\textcolor[rgb]{0.00,0.13,0.44}{##1}}}
\expandafter\def\csname PYG@tok@nn\endcsname{\let\PYG@bf=\textbf\def\PYG@tc##1{\textcolor[rgb]{0.05,0.52,0.71}{##1}}}
\expandafter\def\csname PYG@tok@no\endcsname{\def\PYG@tc##1{\textcolor[rgb]{0.38,0.68,0.84}{##1}}}
\expandafter\def\csname PYG@tok@na\endcsname{\def\PYG@tc##1{\textcolor[rgb]{0.25,0.44,0.63}{##1}}}
\expandafter\def\csname PYG@tok@nb\endcsname{\def\PYG@tc##1{\textcolor[rgb]{0.00,0.44,0.13}{##1}}}
\expandafter\def\csname PYG@tok@nc\endcsname{\let\PYG@bf=\textbf\def\PYG@tc##1{\textcolor[rgb]{0.05,0.52,0.71}{##1}}}
\expandafter\def\csname PYG@tok@nd\endcsname{\let\PYG@bf=\textbf\def\PYG@tc##1{\textcolor[rgb]{0.33,0.33,0.33}{##1}}}
\expandafter\def\csname PYG@tok@ne\endcsname{\def\PYG@tc##1{\textcolor[rgb]{0.00,0.44,0.13}{##1}}}
\expandafter\def\csname PYG@tok@nf\endcsname{\def\PYG@tc##1{\textcolor[rgb]{0.02,0.16,0.49}{##1}}}
\expandafter\def\csname PYG@tok@si\endcsname{\let\PYG@it=\textit\def\PYG@tc##1{\textcolor[rgb]{0.44,0.63,0.82}{##1}}}
\expandafter\def\csname PYG@tok@s2\endcsname{\def\PYG@tc##1{\textcolor[rgb]{0.25,0.44,0.63}{##1}}}
\expandafter\def\csname PYG@tok@vi\endcsname{\def\PYG@tc##1{\textcolor[rgb]{0.73,0.38,0.84}{##1}}}
\expandafter\def\csname PYG@tok@nt\endcsname{\let\PYG@bf=\textbf\def\PYG@tc##1{\textcolor[rgb]{0.02,0.16,0.45}{##1}}}
\expandafter\def\csname PYG@tok@nv\endcsname{\def\PYG@tc##1{\textcolor[rgb]{0.73,0.38,0.84}{##1}}}
\expandafter\def\csname PYG@tok@s1\endcsname{\def\PYG@tc##1{\textcolor[rgb]{0.25,0.44,0.63}{##1}}}
\expandafter\def\csname PYG@tok@gp\endcsname{\let\PYG@bf=\textbf\def\PYG@tc##1{\textcolor[rgb]{0.78,0.36,0.04}{##1}}}
\expandafter\def\csname PYG@tok@sh\endcsname{\def\PYG@tc##1{\textcolor[rgb]{0.25,0.44,0.63}{##1}}}
\expandafter\def\csname PYG@tok@ow\endcsname{\let\PYG@bf=\textbf\def\PYG@tc##1{\textcolor[rgb]{0.00,0.44,0.13}{##1}}}
\expandafter\def\csname PYG@tok@sx\endcsname{\def\PYG@tc##1{\textcolor[rgb]{0.78,0.36,0.04}{##1}}}
\expandafter\def\csname PYG@tok@bp\endcsname{\def\PYG@tc##1{\textcolor[rgb]{0.00,0.44,0.13}{##1}}}
\expandafter\def\csname PYG@tok@c1\endcsname{\let\PYG@it=\textit\def\PYG@tc##1{\textcolor[rgb]{0.25,0.50,0.56}{##1}}}
\expandafter\def\csname PYG@tok@kc\endcsname{\let\PYG@bf=\textbf\def\PYG@tc##1{\textcolor[rgb]{0.00,0.44,0.13}{##1}}}
\expandafter\def\csname PYG@tok@c\endcsname{\let\PYG@it=\textit\def\PYG@tc##1{\textcolor[rgb]{0.25,0.50,0.56}{##1}}}
\expandafter\def\csname PYG@tok@mf\endcsname{\def\PYG@tc##1{\textcolor[rgb]{0.13,0.50,0.31}{##1}}}
\expandafter\def\csname PYG@tok@err\endcsname{\def\PYG@bc##1{\setlength{\fboxsep}{0pt}\fcolorbox[rgb]{1.00,0.00,0.00}{1,1,1}{\strut ##1}}}
\expandafter\def\csname PYG@tok@mb\endcsname{\def\PYG@tc##1{\textcolor[rgb]{0.13,0.50,0.31}{##1}}}
\expandafter\def\csname PYG@tok@ss\endcsname{\def\PYG@tc##1{\textcolor[rgb]{0.32,0.47,0.09}{##1}}}
\expandafter\def\csname PYG@tok@sr\endcsname{\def\PYG@tc##1{\textcolor[rgb]{0.14,0.33,0.53}{##1}}}
\expandafter\def\csname PYG@tok@mo\endcsname{\def\PYG@tc##1{\textcolor[rgb]{0.13,0.50,0.31}{##1}}}
\expandafter\def\csname PYG@tok@kd\endcsname{\let\PYG@bf=\textbf\def\PYG@tc##1{\textcolor[rgb]{0.00,0.44,0.13}{##1}}}
\expandafter\def\csname PYG@tok@mi\endcsname{\def\PYG@tc##1{\textcolor[rgb]{0.13,0.50,0.31}{##1}}}
\expandafter\def\csname PYG@tok@kn\endcsname{\let\PYG@bf=\textbf\def\PYG@tc##1{\textcolor[rgb]{0.00,0.44,0.13}{##1}}}
\expandafter\def\csname PYG@tok@o\endcsname{\def\PYG@tc##1{\textcolor[rgb]{0.40,0.40,0.40}{##1}}}
\expandafter\def\csname PYG@tok@kr\endcsname{\let\PYG@bf=\textbf\def\PYG@tc##1{\textcolor[rgb]{0.00,0.44,0.13}{##1}}}
\expandafter\def\csname PYG@tok@s\endcsname{\def\PYG@tc##1{\textcolor[rgb]{0.25,0.44,0.63}{##1}}}
\expandafter\def\csname PYG@tok@kp\endcsname{\def\PYG@tc##1{\textcolor[rgb]{0.00,0.44,0.13}{##1}}}
\expandafter\def\csname PYG@tok@w\endcsname{\def\PYG@tc##1{\textcolor[rgb]{0.73,0.73,0.73}{##1}}}
\expandafter\def\csname PYG@tok@kt\endcsname{\def\PYG@tc##1{\textcolor[rgb]{0.56,0.13,0.00}{##1}}}
\expandafter\def\csname PYG@tok@sc\endcsname{\def\PYG@tc##1{\textcolor[rgb]{0.25,0.44,0.63}{##1}}}
\expandafter\def\csname PYG@tok@sb\endcsname{\def\PYG@tc##1{\textcolor[rgb]{0.25,0.44,0.63}{##1}}}
\expandafter\def\csname PYG@tok@k\endcsname{\let\PYG@bf=\textbf\def\PYG@tc##1{\textcolor[rgb]{0.00,0.44,0.13}{##1}}}
\expandafter\def\csname PYG@tok@se\endcsname{\let\PYG@bf=\textbf\def\PYG@tc##1{\textcolor[rgb]{0.25,0.44,0.63}{##1}}}
\expandafter\def\csname PYG@tok@sd\endcsname{\let\PYG@it=\textit\def\PYG@tc##1{\textcolor[rgb]{0.25,0.44,0.63}{##1}}}

\def\PYGZbs{\char`\\}
\def\PYGZus{\char`\_}
\def\PYGZob{\char`\{}
\def\PYGZcb{\char`\}}
\def\PYGZca{\char`\^}
\def\PYGZam{\char`\&}
\def\PYGZlt{\char`\<}
\def\PYGZgt{\char`\>}
\def\PYGZsh{\char`\#}
\def\PYGZpc{\char`\%}
\def\PYGZdl{\char`\$}
\def\PYGZhy{\char`\-}
\def\PYGZsq{\char`\'}
\def\PYGZdq{\char`\"}
\def\PYGZti{\char`\~}
% for compatibility with earlier versions
\def\PYGZat{@}
\def\PYGZlb{[}
\def\PYGZrb{]}
\makeatother

\renewcommand\PYGZsq{\textquotesingle}

\begin{document}

\maketitle
\tableofcontents
\phantomsection\label{index::doc}


Contents:


\chapter{Manual for rt2detection package.}
\label{intro:welcome-to-rt2detection-s-documentation}\label{intro:manual-for-rt2detection-package}\label{intro::doc}\begin{quote}

Calum Chamberlain, SGEES, Victoria University of Wellington, PO Box 600, Wellington 6140\textbackslash{}calum.chamberlain@vuw.ac.nz\textbackslash{}
\end{quote}


\section{Introduction}
\label{intro:introduction}
This manual is intended as a user guide for the rt2detection package, written in python, for
the routine processing of seismic data collected on reftek RT130 dataloggers.  This package
is fairly minimal and rather than re-doing a lot of processing steps calls upon external
routines that need to be installed seperately (this is covered in the installation section).

The heart of this processing flow is that we want to simply take raw data through to triggered,
multiplexed miniseed files which can be read in by seisan.  This processing flow also includes
automatic picking capability, although this is by no means optimized and the parameters for this
picker should be optimized for the network you are working with.  The same can be said of the
parameters used in the STA-LTA detection routine.


\section{Installation}
\label{intro:installation}
As the main linking code is written in python, there is little compliation to be done, however
this package does require the following to be installed on your machine (according to each
programs install instructions):
\begin{quote}

Pascal tools (\href{http://www.passcal.nmt.edu/content/software-resources}{http://www.passcal.nmt.edu/content/software-resources})
Obspy (\href{https://github.com/obspy/obspy/wiki}{https://github.com/obspy/obspy/wiki})
\end{quote}

Further to this, to use the automated picker you will need to compile the C code found
in the emph\{picker\} directory if this distribution, for this you will need a C compiler
(gcc has been tested and works well, nothing else has been tested).  Once you have
installed a C compiler you should change the variables in both picker/src/Makefile and
picker/libmseed/Makefile to the path of your C compiler (or, if you have installed it gcc
correctly, just to gcc).  You can then compile the picker by typing, when in the picker
directory:
\textgreater{}make clean
\textgreater{}make rtquake
\textgreater{}cp rdtrigL ../.

You should also run the seisan commands:
\textgreater{}remodl
\textgreater{}setbrn
to set the local travel time tables required for the hypo71 location algorithm.

If you have not generated the seisan databases that you want the data to be stored in
(in the REA and WAV directories of seisan) you should run:\textbackslash{}
\textgreater{}makerea\textbackslash{}
to generate them before running any of the codes.

Now you should be ready to roll.


\section{Structure}
\label{intro:structure}\begin{description}
\item[{This distribution contains five directories:}] \leavevmode
doc
picker
RT\_data
MS\_data
Merged\_data

\end{description}

The doc directory contains this document about the package.  The picker directory contains
the source code and libraries for the picking routine.  The RT\_data directory is where
you should upload your raw reftek data to once you have run it through the pascal tools,
neo software. emph\{This must be uploaded in station directories, e.g. RT\_data/WTSZ/misc.zip\}.
The MS\_data directory will be the location of the converted, but un-merged
miniseed data.  Merged\_data will contain all of the multiplexed miniseed files, these will
all later be cleaned when running the programs.


\section{Python routines}
\label{intro:python-routines}
rt2detection.py is the overarching routine which calls all the others.  This will
take your data in steps from raw to triggered, picked data in the seisan database.
The top of this file should be editted to include your parameters and network
information.  All parameters are explained in the file. Around line 71 is the network
information which needs to be adjusted for your network.

rt2detection calls upon STALTA.py which currently runs the convenience STALTA methods
within obspy - the parameters for this routine can be adjusted within this code, and
again are commented as to what they do.

The final python routine, makeSfile.py simply builds empty nordic files which are then
read in by the picking routine.

Finally all files are moved to the seisan database prescribed in the rt2detection parameters.

Parameters for the picker are set in the rtquake.par file.


\chapter{Main}
\label{modules:main}\label{modules::doc}

\section{rt2detection}
\label{modules:rt2detection}\label{modules:module-rt2detection}\index{rt2detection (module)}
Code designed to take data downloded using defaults.neo and located in the RT\_data
folder of this ditribution and convert this to miniseed, multiplex it and run
seisan's continuous detection, sta/lta routine over this data.

ver 0.1 - Converts and defaults.merges data, but only for one day

ver 0.2 - Converts and defaults.merges data for multiple days
\begin{description}
\item[{ver 1.1 - Converts and defaults.merges data for multiple days and can run a python}] \leavevmode
detection routine (STA/LTA) over the data

\item[{ver 2.1 - Uses the filterdefaults.picker routine to automatically pick data, parameters}] \leavevmode
for this are not yet optimized

\end{description}

07/11/14 - Calum Chamberlain - Victoria University of Wellington, NZ


\chapter{Pro}
\label{modules:pro}
Codes to run main functions.


\section{Sfile\_util}
\label{modules:module-Sfile_util}\label{modules:sfile-util}\index{Sfile\_util (module)}
Part of the rt2detection module to read nordic format s-files and write them
EQcorrscan is a python module designed to run match filter routines for
seismology, within it are routines for integration to seisan and obspy.
With obspy integration (which is necessary) all main waveform formats can be
read in and output.

Code generated by Calum John Chamberlain of Victoria University of Wellington,
2015.

All rights reserved.
\index{PICK (class in Sfile\_util)}

\begin{fulllineitems}
\phantomsection\label{modules:Sfile_util.PICK}\pysiglinewithargsret{\strong{class }\code{Sfile\_util.}\bfcode{PICK}}{\emph{station}, \emph{channel}, \emph{impulsivity}, \emph{phase}, \emph{weight}, \emph{polarity}, \emph{time}, \emph{coda}, \emph{amplitude}, \emph{peri}, \emph{azimuth}, \emph{velocity}, \emph{AIN}, \emph{SNR}, \emph{azimuthres}, \emph{timeres}, \emph{finalweight}, \emph{distance}, \emph{CAZ}}{}
Pick information for seisan implimentation

\end{fulllineitems}

\index{readpicks() (in module Sfile\_util)}

\begin{fulllineitems}
\phantomsection\label{modules:Sfile_util.readpicks}\pysiglinewithargsret{\code{Sfile\_util.}\bfcode{readpicks}}{\emph{sfilename}}{}
Function to read pick informaiton from the s-file
\begin{quote}\begin{description}
\item[{Returns}] \leavevmode
Sfile\_tile.PICK

\end{description}\end{quote}

\end{fulllineitems}

\index{readwavename() (in module Sfile\_util)}

\begin{fulllineitems}
\phantomsection\label{modules:Sfile_util.readwavename}\pysiglinewithargsret{\code{Sfile\_util.}\bfcode{readwavename}}{\emph{sfilename}}{}
Convenience function to extract the waveform filename from the s-file,
returns a list of waveform names found in the s-file as multiples can
be present.

\end{fulllineitems}

\index{blanksfile() (in module Sfile\_util)}

\begin{fulllineitems}
\phantomsection\label{modules:Sfile_util.blanksfile}\pysiglinewithargsret{\code{Sfile\_util.}\bfcode{blanksfile}}{\emph{wavefile}, \emph{evtype}, \emph{userID}, \emph{outdir}, \emph{overwrite}}{}~\begin{quote}

Module to generate an empty s-file with a populated header for a given
waveform.
\end{quote}
\begin{quote}

\# Arguments are the path of a wavefile (multiplexed miniseed file required)
\# Event type (L,R,D) and user ID (four characters as used in seisan)
\end{quote}
\begin{quote}

\# Example s-file format:
\# 2014  719  617 50.2 R                                                         1
\# ACTION:ARG 14-11-11 10:53 OP:CALU STATUS:               ID:20140719061750     I
\# 2014/07/2014-07-19-0617-50.SAMBA\_030\_00                                       6
\# STAT SP IPHASW D HRMM SECON CODA AMPLIT PERI AZIMU VELO AIN AR TRES W  DIS CAZ7
\end{quote}

\end{fulllineitems}

\index{populateSfile() (in module Sfile\_util)}

\begin{fulllineitems}
\phantomsection\label{modules:Sfile_util.populateSfile}\pysiglinewithargsret{\code{Sfile\_util.}\bfcode{populateSfile}}{\emph{sfilename}, \emph{picks}}{}
Module to populate a blank nordic format S-file with pick information,
arguments required are the filename of the blank s-file and the picks
where picks is a dictionary of picks including station, channel,
impulsivity, phase, weight, polarity, time, coda, amplitude, peri, azimuth,
velocity, SNR, azimuth residual, Time-residual, final weight,
epicentral distance \& azimuth from event to station.

This is a full pick line information from the seisan manual, P. 341

\end{fulllineitems}



\section{py\_picker}
\label{modules:module-py_picker}\label{modules:py-picker}\index{py\_picker (module)}
Python automated seismic picker implimentations for the rt2detection package.
\index{picker\_plot() (in module py\_picker)}

\begin{fulllineitems}
\phantomsection\label{modules:py_picker.picker_plot}\pysiglinewithargsret{\code{py\_picker.}\bfcode{picker\_plot}}{\emph{stream}, \emph{picks}, \emph{types}, \emph{show=True}, \emph{savefile='`}}{}
Plotting fucntion for the picker routine - this will pot the waveforms and
the picks to allow for pick veriication.
\begin{quote}\begin{description}
\end{description}\end{quote}

\end{fulllineitems}

\index{seism\_picker() (in module py\_picker)}

\begin{fulllineitems}
\phantomsection\label{modules:py_picker.seism_picker}\pysiglinewithargsret{\code{py\_picker.}\bfcode{seism\_picker}}{\emph{picktype}, \emph{args}, \emph{stream}}{}
A simple cover function for th eobspy picking modules
\begin{quote}\begin{description}
\item[{Parameters}] \leavevmode\begin{itemize}
\item {} 
\textbf{\texttt{picktype}} (\emph{String}) -- Eother Baer or AR for Auto-regressive

\item {} 
\textbf{\texttt{args}} (\emph{Class}) -- List of appropriate arguments for the chosen pick type -
defined in picker\_par.py

\end{itemize}

\item[{Returns}] \leavevmode
p\_pick, s\_pick

\end{description}\end{quote}

\end{fulllineitems}



\section{STALTA}
\label{modules:stalta}\label{modules:module-STALTA}\index{STALTA (module)}

\chapter{Par}
\label{modules:par}
User-editable codes to input parameters for \emph{core} files.
Scripts are coded in the files with a full description of the parameters


\section{overallpars}
\label{modules:overallpars}\label{modules:module-overallpars}\index{overallpars (module)}
\# A python style declaration of variables used in rt2detection.py
\index{STATION (class in overallpars)}

\begin{fulllineitems}
\phantomsection\label{modules:overallpars.STATION}\pysiglinewithargsret{\strong{class }\code{overallpars.}\bfcode{STATION}}{\emph{name}, \emph{netcode}, \emph{loccode}, \emph{das}, \emph{ch1}, \emph{ch2}, \emph{ch3}}{}
Station information for seismic station
\index{stacount (overallpars.STATION attribute)}

\begin{fulllineitems}
\phantomsection\label{modules:overallpars.STATION.stacount}\pysigline{\bfcode{stacount}\strong{ = 0}}
\end{fulllineitems}


\end{fulllineitems}



\section{picker\_par}
\label{modules:module-picker_par}\label{modules:picker-par}\index{picker\_par (module)}
Python imput arguments for the python auto-pickers
\index{AR\_ARGS (class in picker\_par)}

\begin{fulllineitems}
\phantomsection\label{modules:picker_par.AR_ARGS}\pysiglinewithargsret{\strong{class }\code{picker\_par.}\bfcode{AR\_ARGS}}{\emph{f1}, \emph{f2}, \emph{lta\_p}, \emph{sta\_p}, \emph{lta\_s}, \emph{sta\_s}, \emph{m\_p}, \emph{m\_s}, \emph{l\_p}, \emph{l\_s}, \emph{s\_pick}}{}
Input arguments for the AR picker

\end{fulllineitems}

\index{BAER\_ARGS (class in picker\_par)}

\begin{fulllineitems}
\phantomsection\label{modules:picker_par.BAER_ARGS}\pysiglinewithargsret{\strong{class }\code{picker\_par.}\bfcode{BAER\_ARGS}}{\emph{tdownmax}, \emph{tupevent}, \emph{thr1}, \emph{thr2}, \emph{preset\_len}, \emph{p\_dur}}{}
Input arguments for the Baer picker

\end{fulllineitems}



\section{trigger\_par}
\label{modules:trigger-par}\label{modules:module-trigger_par}\index{trigger\_par (module)}
Declaration of variables to be used as defaults in the STALTA.py
routine.


\renewcommand{\indexname}{Python Module Index}
\begin{theindex}
\def\bigletter#1{{\Large\sffamily#1}\nopagebreak\vspace{1mm}}
\bigletter{o}
\item {\texttt{overallpars}}, \pageref{modules:module-overallpars}
\indexspace
\bigletter{p}
\item {\texttt{picker\_par}}, \pageref{modules:module-picker_par}
\item {\texttt{py\_picker}}, \pageref{modules:module-py_picker}
\indexspace
\bigletter{r}
\item {\texttt{rt2detection}}, \pageref{modules:module-rt2detection}
\indexspace
\bigletter{s}
\item {\texttt{Sfile\_util}}, \pageref{modules:module-Sfile_util}
\item {\texttt{STALTA}}, \pageref{modules:module-STALTA}
\indexspace
\bigletter{t}
\item {\texttt{trigger\_par}}, \pageref{modules:module-trigger_par}
\end{theindex}

\renewcommand{\indexname}{Index}
\printindex
\end{document}
