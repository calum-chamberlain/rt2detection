\documentclass[a4paper,12pt,final]{article}
\usepackage[T1]{fontenc}
\usepackage[utf8]{inputenc}
\usepackage{lmodern,graphicx,textcomp}
\usepackage[top=1.2in, left=1.25in, bottom=1.2in, right=1.25in]{geometry}
\usepackage[bookmarks]{hyperref}
\usepackage{sidecap,rotating}
\usepackage{wrapfig,caption,subcaption}

%\pagestyle{fancy}

\begin{document}

\begin{center}
  \textbf{Manual for rt2detection package.}\\
   \vspace*{1\baselineskip}
  Calum Chamberlain, SGEES, Victoria University of Wellington, PO Box 600, Wellington 6140\\calum.chamberlain@vuw.ac.nz\\
  \vspace*{1\baselineskip}
  
\end{center} %\vspace*{1\baselineskip}

\section{Introduction}

This manual is intended as a user guide for the rt2detection package, written in python, for
the routine processing of seismic data collected on reftek RT130 dataloggers.  This package
is fairly minimal and rather than re-doing a lot of processing steps calls upon external
routines that need to be installed seperately (this is covered in the installation section).

The heart of this processing flow is that we want to simply take raw data through to triggered,
multiplexed miniseed files which can be read in by seisan.  This processing flow also includes
automatic picking capability, although this is by no means optimized and the parameters for this
picker should be optimized for the network you are working with.  The same can be said of the
parameters used in the STA-LTA detection routine.

\section{Installation}

As the main linking code is written in python, there is little compliation to be done, however
this package does require the following to be installed on your machine (according to each
programs install instructions):
\begin{itemize}
	\item Pascal tools (http://www.passcal.nmt.edu/content/software-resources)
	\item Obspy (https://github.com/obspy/obspy/wiki)
\end{itemize}

Further to this, to use the automated picker you will need to compile the C code found
in the \emph{picker} directory if this distribution, for this you will need a C compiler
(gcc has been tested and works well, nothing else has been tested).  Once you have
installed a C compiler you should change the variables in both picker/src/Makefile and 
picker/libmseed/Makefile to the path of your C compiler (or, if you have installed it gcc
correctly, just to gcc).  You can then compile the picker by typing, when in the picker
directory:\\
>make clean\\
>make rtquake\\
>cp rdtrigL ../.\\

You should also run the seisan commands:\\
>remodl\\
>setbrn\\
to set the local travel time tables required for the hypo71 location algorithm.

If you have not generated the seisan databases that you want the data to be stored in
(in the REA and WAV directories of seisan) you should run:\\
>makerea\\
to generate them before running any of the codes.

Now you should be ready to roll.

\section{Structure}

This distribution contains five directories:
\begin{itemize}
	\item doc
	\item picker
	\item RT\_data
	\item MS\_data
	\item Merged\_date
\end{itemize}

The doc directory contains this document about the package.  The picker directory contains
the source code and libraries for the picking routine.  The RT\_data directory is where
you should upload your raw reftek data to once you have run it through the pascal tools, 
neo software. \emph{This must be uploaded in station directories, e.g. RT\_data/WTSZ/misc.zip}.
The MS\_data directory will be the location of the converted, but un-merged
miniseed data.  Merged\_data will contain all of the multiplexed miniseed files, these will
all later be cleaned when running the programs.

\section{Python routines}

There are three python routines included in this package:
\begin{itemize}
	\item{makeSfile.py}
	\item{STALTA.py}
	\item{rt2detection.py}
\end{itemize}

rt2detection.py is the overarching routine which calls all the others.  This will
take your data in steps from raw to triggered, picked data in the seisan database.
The top of this file should be editted to include your parameters and network 
information.  All parameters are explained in the file. Around line 71 is the network
information which needs to be adjusted for your network.

rt2detection calls upon STALTA.py which currently runs the convenience STALTA methods
within obspy - the parameters for this routine can be adjusted within this code, and
again are commented as to what they do.

The final python routine, makeSfile.py simply builds empty nordic files which are then
read in by the picking routine.

Finally all files are moved to the seisan database prescribed in the rt2detection parameters.

Parameters for the picker are set in the rtquake.par file.

\end{document}
